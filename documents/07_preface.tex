\chapter*{PREFACE}
    \def\baselinestretch{1.0}
    This work presents an investigation into the distinct class of high-energy celestial objects known as Supersoft X-ray Sources (SSS). Through spectral fitting techniques and variability analysis, the work aims to study the physical characteristics and mechanisms driving these luminous sources, often involving accreting white dwarfs within binary systems. This study draws upon data from several high-resolution space observatories, to develop robust continuum spectral models using data from different epochs to provide reasonable fits to available data from a variety of types of SSS. The study also reports three distinct variabilities in the lightcurve of a specific enigmatic SSS with data from two different observatories.
    
    \textit{Chapter 1} introduces the field of X-ray astronomy and the progression from early detectors to modern missions which provide detailed data for SSS studies. It discusses the key models for SSS, their classifications, and the role of companion stars in these systems. This chapter also defines the research problem and outlines the objectives that guide the study. \textit{Chapter 2} explains the methodology used to analyse SSS, covering spectral fitting and the application of the Lomb-Scargle periodogram for detecting periodic signals. It provides a framework for data acquisition, preprocessing, and statistical analysis, laying the foundation for the subsequent chapters.
    
    \textit{Chapter 3} focuses on the spectral characteristics of RX J0925.7-4758, a notable SSS observed over 25 years across multiple observatories. The chapter develops a non-local thermodynamic equilibrium (NLTE) model that captures the continuum spectrum, revealing insights into the astrophysical properties of RX J0925.7-4758, including its effective temperature and potential accretion mechanisms. In \textit{Chapter 4}, the high-resolution X-ray spectra of RX J0925.7-4758 obtained from the XMM-Newton observatory's RGS instrument are analysed. A refined model, incorporating interstellar medium absorption and additional emission components, provides a more accurate depiction of the X-ray emission processes involved, highlighting improvements over previous models.
    
    \textit{Chapter 5} details the development of a Python-based tool that overlays atomic data from the NIST Atomic Spectra Database onto X-ray spectra, facilitating line identification for SSS. This tool enhances preliminary analysis, supporting astronomers in their study of spectral line features and the elemental composition of SSS.
    
    \textit{Chapter 6} presents an analysis of a broader dataset of SSS, exploring variations in spectral properties, luminosity, and variability characteristics across sources in the Milky Way and the Large Magellanic Cloud galaxies. Notable findings include differences in interstellar absorption, correlations between luminosity and mass accretion rates, and the identification of periodic modulations in lightcurves, providing insights into environmental influences on SSS.
    
    The final chapter, \textit{Chapter 7}, summarises the findings of this work, discussing their implications for the evolution of SSS and identifying future research directions. The chapter suggests further observational studies, high-resolution timing, and advanced theoretical models to deepen our understanding of these complex systems.