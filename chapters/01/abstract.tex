\newpage
\begin{center}
    	%\emph{Abstract of chapter \ref{chap:introduction}}
    	\emph{Abstract}
    \end{center}
    This chapter introduces the field of X-ray astronomy, highlighting its significance in studying high-energy celestial phenomena. It covers the advancements from early X-ray detectors to contemporary missions like Chandra and XMM-Newton, which have provided detailed spectral data of various X-ray sources. A special focus is on luminous supersoft X-ray sources (SSS), a distinct category identified by their exceptionally soft spectra and high luminosities. The chapter explores the prevailing models for SSS, primarily involving accreting white dwarfs in binary systems, and their potential as progenitors of Type Ia supernovae. It also delves into the classifications and companion star types within SSS systems, and the significance of understanding classical novae in the context of X-ray observations. Finally, it outlines the research problem, hypothesis, and objectives, aiming to develop robust models for SSS spectra and understand their underlying physical mechanisms.