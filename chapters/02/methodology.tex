\chapter{METHODOLOGY} \label{chap:methodology}
    %\doublespacing
    \minitoc
    
    \newpage
    \begin{center}
    	\emph{Abstract of chapter \ref{chap:methodology}}
    \end{center}
    
    Insert abstract here.
    
    \section{Spectral Study of Supersoft X-ray Sources} \label{methodology:spectral}
    	
    	\subsection{Basics of Spectral Curve Fitting}
    		Spectral curve fitting is a fundamental technique in data analysis, particularly in fields involving spectroscopy. It involves modeling observed spectral data using mathematical functions, often with adjustable parameters, to extract meaningful information about the underlying physical processes.

			\subsubsection{The Spectral Fitting Problem}
				Spectrometers do not directly measure the true spectrum of a source. Instead, they record photon counts in specific instrument channels, which are influenced by the instrument's response function. Mathematically, the observed spectrum $C(I)$ is related to the actual spectrum $f(E)$ through the integral equation (\ref{eqn:obs-spec}).
				\begin{align}
					C(I)=\int_0^\infty{f(E)R(I,E)\diff{E}} \label{eqn:obs-spec}
				\end{align}
				In equation (\ref{eqn:obs-spec}), $R(I,E)$ represents the instrument response function, which describes how photons of different energies are detected and recorded in the instrument channels. Inverting this equation to directly obtain the true spectrum from the observed data can be challenging due to non-uniqueness and instability issues. Therefore, an alternative approach of model selection and fitting is commonly adopted.
			
			\subsubsection{Model-Based Fitting}
				In spectral curve fitting, the initial step involves selecting a suitable model spectrum that can adequately represent the observed data. This model is often described by a set of parameters, and is of the form $f(E,p_1,p_2,p_3,\dots)$. These parameters act as adjustable variables that can be modified to fine-tune the shape and characteristics of the model spectrum.
				
				The process of parameter adjustment involves modifying the values of $p_1,p_2,p_3,\dots$ in an iterative manner to minimize the discrepancy between the model spectrum and the observed data. This is essentially a trial-and-error approach where different combinations of parameter values are explored until a satisfactory fit is achieved. The goal is to find the optimal set of parameters that produces a model spectrum that closely aligns with the observed data, capturing the underlying physical processes or phenomena.
			
			\subsubsection{Fitting Methodology}
				Once a suitable model spectrum is chosen and its parameters are defined, the next step involves calculating a predicted count spectrum, denoted as $C_P(I)$. This predicted spectrum is generated by applying the chosen model to the instrument response function and integrating over the energy range. The resulting $C_P(I)$ represents the expected count spectrum based on the model and the instrument's characteristics.

To assess the agreement between the predicted count spectrum and the observed data, a statistical metric known as the fit statistic is employed. The $\chi^2$ statistic is a commonly used fit statistic, which quantifies the deviation between the two spectra. It is calculated as the sum of the squared differences between the observed and predicted counts, normalized by the variance of the observed counts:
				\begin{align}
					\chi^2\equiv\sum{\dfrac{[C(I)-C_P(I)]^2}{(\sigma(I))^2}} \label{eqn:chi-sq}
				\end{align}
				where $C(I)$ is the observed count in channel $I$, $C_P(I)$ is the predicted count in channel $I$, and $\sigma(I)$ is the estimated error in channel $I$. The error term is often approximated as the square root of the observed count, assuming Poisson statistics.
				
				The goal of spectral curve fitting is to find the set of model parameters that minimizes the $\chi^2$ value. This indicates a better agreement between the predicted and observed spectra, suggesting that the chosen model and its parameters accurately describe the underlying physical processes. To achieve this, the parameters are systematically varied and the corresponding $\chi^2$ values are calculated. The set of parameters that yields the lowest $\chi^2$ value is considered the best fit. This iterative process continues until a satisfactory minimum $\chi^2$ value is reached.
			
			\subsubsection{Assessing the Goodness of Fit}
			
			\subsubsection{Error Estimation}
%    		\begin{itemize}
%				\item Spectrometer does not really obtain the spectrum from a source
%				\item Actually obtains the photon counts $C$ in specific instrument channels $I$
%				\item Suppose
%			
%				\quad\quad$f(E)$: actual energy spectrum
%			
%				\quad\quad$R(I,E)$: instrument response
%				\item Observed spectrum $C(I)$ is related to the actual spectrum $f(E)$ as:
%				\begin{equation} \label{c-e}
%				C(I)=\int_0^\infty{f(E)R(I,E)\mathrm{d}E}
%				\end{equation}
%				\item Inversion of equation (\ref{c-e}) tends to be non-unique and unstable
%			\end{itemize}
%			
%			The alternative:
%			\begin{itemize}
%				\item Choose a model spectrum described in terms of a few parameters
%				$$f(E,p_1,p_2,p_3,\dots)$$
%				\item Then adjust the parameters $p_1,p_2,p_3,\dots$ to fit the model to the data\\[0.8cm]
%			
%				\quad\quad\quad A game of trial and error!
%			\end{itemize}
%			
%			\begin{itemize}
%				\item For each $f(E)$, a predicted count spectrum $C_P(I)$ is calculated
%				\item A fit statistic is computed to judge whether the model spectrum fits the data obtained
%				\item Parameters are varied to obtain the best fit model $f_B(E)$
%				\item Most common fit statistic: $\chi^2$
%				$$\chi^2\equiv\sum{\dfrac{[C(I)-C_P(I)]^2}{(\sigma(I))^2}}$$
%				\quad\quad\quad where $\sigma(I)$: error in channel $I$; usually estimated by $\sqrt{C(I)}$
%			\end{itemize}
%			
%			$$\chi^2=\sum{\dfrac{[C(I)-C_P(I)]^2}{(\sigma(I))^2}}$$
%			What values of $\chi^2$ correspond to the \emph{best fit}?
%			\begin{itemize}
%				\item Degrees of freedom
%			
%				$\nu\equiv$ (number of channels) $-$ (number of model parameters)
%				\item Reduced $\chi^2$: $\dfrac{\chi^2}{\nu}$
%				\item Goodness-of-fit criterion: when $\chi^2$ value is close to $\nu$, i.e.,
%			
%				\quad\quad\quad\quad\quad$\chi^2\sim\nu$\quad or\quad $\dfrac{\chi^2}{\nu}\sim 1$
%			\end{itemize}
%			
%			\begin{itemize}
%				\item $\dfrac{\chi^2}{\nu}\sim 1$\quad$\Rightarrow$\quad A good fit
%				\item $\dfrac{\chi^2}{\nu}\gg 1$\quad$\Rightarrow$\quad A poor fit
%				\item $\dfrac{\chi^2}{\nu}\ll 1$\quad$\Rightarrow$\quad Errors on data overestimated\\[0.5cm]
%				\item Error width: range of values within which one can be confident of the true value of a parameter
%			\end{itemize}
			
    	\subsection{Software Implementation}
    		Main components for spectral fitting:\\[0.1cm]
			\begin{itemize}
				\item $D(I)$: Observed spectra\\[0.1cm]
				\item $B(I)$: Background measurements\\[0.1cm]
				\item $R(I,E)$: Corresponding instrument response\\[0.1cm]
				\item $M(E)$: Set of model spectra\\[0.1cm]
			\end{itemize}
			
			\begin{itemize}
				\item $C(I)$ is obtained from $D(I)$ and $B(I)$ as:
				\begin{equation} \label{C-I}
				C(I)=\dfrac{D(I)}{a_Dt_D}-\dfrac{b_D}{b_B}\dfrac{B(I)}{a_B t_B}
				\end{equation}
				\quad\quad\quad$a_D,a_B$: area scaling factors
			
				\quad\quad\quad$b_D,b_B$: background scaling factors
			
				\quad\quad\quad$t_D,t_B$: exposure times
			\end{itemize}
			
			\begin{itemize}
				\item $R(I,E)$ is a continuous function
				\item Made discrete by defining energy ranges $E_j$ as:
				\begin{equation} \label{R_D-Ij}
				R_D(I,j)=\dfrac{1}{E_j-E_{j-1}}{\int_{E_{j-1}}^{E_j}{R(I,E)\mathrm{d}E}}
				\end{equation}
				\item Optional: Multiplication of an auxiliary response array $A_D(j)$ into $R_D(I,j)$
				\begin{equation*}
				R_D(I,j)\quad\rightarrow\quad R_D(I,j)\cdot A_D(j)
				\end{equation*}
				\item This represents the \emph{redistribution matrix function} (RMF)
			\end{itemize}
			
			\begin{itemize}
				\item The model spectrum array is calculated as
				\begin{equation} \label{M_D-j}
				M_D(j)=\int_{E_{j-1}}^{E_j}{M(E)\mathrm{d}E}
				\end{equation}
			\end{itemize}
			
			\begin{itemize}
				\item $C(I)$\quad$\leftarrow$\quad$D(I)$ \& $B(I)$
				\item $R_D(I,j)$\quad$\leftarrow$\quad$R(I,E)$
				\item $M_D(j)$\quad$\leftarrow$\quad$M(E)$\\[0.5cm]
			
				\item The predicted count spectrum $C_P(I)$ is now obtained as
				\begin{equation} \label{C_P-I}
				C_P(I)=R_D(I,j)\cdot M_D(j)
				\end{equation}\\[0.5cm]
				\item The $\chi^2$ for this particular fit is obtained as
				\begin{equation} \label{chi-2}
				\chi^2=\sum_I{[C(I)-C_P(I)]^2}
				\end{equation}
			\end{itemize}
			
			\begin{itemize}
				\item More generally...\\[0.5cm]
			
				Given a set of spectra $\mathbb{C}(I)$, each supplied as a function of detector channels, a set of theoretical models {$\mathbb{M}(E)_j$}, each expressed in terms of a vector of energies together with a set of functions {$\mathbb{R}(I,E)_j$} that map the channels to the energies, minimize an objective function $S$ of $\mathbb{C}$, {$\mathbb{R}(I,E)_j$}, {$\mathbb{M}(E)_j$} using a fitting algorithm, i.e.
			
				$$S=S(\mathbb{C},\sum_k{\mathbb{M}_j^k\cdot\mathbb{R}_j^k})$$\\[0.5cm]
			
				\item Specifically, using the \emph{Levenberg-Marquardt algorithm}, $S$ reduces to $\chi^2$
			
				$$S=\chi^2=\sum_i{(C_i-M_iR_i)^2}$$
			\end{itemize}
			
    \section{Variability Study of Supersoft X-ray Sources} \label{methodology:variability}
    
    	\subsection{Introduction to Lomb-Scargle Periodogram}
    		%Introduce the concept of periodograms, emphasizing their role in time series analysis for detecting periodic signals.
    		Periodograms are a versatile tool for identifying and quantifying periodic patterns within data. By transforming a time series into the frequency domain, periodograms reveal the frequencies at which the signal exhibits significant power. This spectral analysis is invaluable for detecting periodic components that might be obscured in the original time domain representation.
    		
    		The height of peaks in a periodogram provides an estimate of the amplitude of the corresponding periodic component. This allows for a quantitative assessment of the strength of periodic signals. Additionally, the frequency of these peaks directly indicates the periodicity of the underlying patterns.
    		
    		While traditional periodograms are well-suited for uniformly sampled data, their applicability can be limited when dealing with irregularly spaced time series. In such scenario, the Lomb-Scargle periodogram emerges as a viable alternative \cite{lomb1976least,scargle1982studies}. Specifically designed to handle unevenly spaced data, the Lomb-Scargle periodogram accounts for gaps in the time series, providing a robust and reliable method for detecting periodic signals.
    		
    		%\subsubsection{Historical Background}
    		
    		%\subsubsection{Importance in Astrophysics}
    	
    	\subsection{Theory of Lomb-Scargle Periodogram}
    	
    		\subsubsection{Uniform and non-uniform sampling of data}
    			\paragraph{Uniform sampling of data:}
    			Suppose that $g(t)$ is a \textit{signal function} and $W(t)$ represents the observation \textit{window function}, then the observed signal is represented by the following function:
    			\begin{align}
    				g_\text{obs}(t)=g(t)W(t) \label{eqn:obs-func}
    			\end{align}
    			Then, by the Fourier convolution theorem, we have,
    			\begin{align}
    				\mathscr{F}\left\lbrace g_\text{obs} \right\rbrace = \mathscr{F}\left\lbrace g(t) \right\rbrace * \mathscr{F}\left\lbrace W(t) \right\rbrace \label{eqn:FT-obs-func}
    			\end{align}
    			In the case of the discrete Fourier transform, where the continuous signal is sampled at regular intervals, the window function becomes a Dirac comb, which is a regular grid of Dirac delta functions with spacing of $\Delta t$, i.e.
    			\begin{align}
    				W(t)=\text{III}_{\Delta t}(t)=\sum_{n=-\infty}^{\infty}{\delta(t-n\Delta t)} \label{eqn:dirac-comb}
    			\end{align}
    			
    			Hence, equation \ref{eqn:obs-func} becomes
    			\begin{align}
    				g_\text{obs}(t)=g(t)\text{III}_{\Delta t}(t) \label{eqn:obs-func:dft}
    			\end{align}
    			Taking the Fourier transform of equation (\ref{eqn:obs-func:dft}), we can have
    			\begin{align}
    				\hat{g}_\text{obs}(f)&=\FT{g_\text{obs}(t)}{f} \nonumber \\
    					&=\FT{g(t)\text{III}_{\Delta t}(t)}{f} \nonumber \\
    					&=\FT{g(t)\left\lbrace \sum_{n=-\infty}^{\infty}{\delta(t-n\Delta t)} \right\rbrace}{f} \nonumber \\
    					&=\sum_{n=-\infty}^{\infty}{\int_{-\infty}^{\infty}{\delta(t-n\Delta t)g(t)e^{-2\pi ift}\diff{t}}} \nonumber \\
    				\implies\hat{g}_\text{obs}(f)&=\sum_{n=-\infty}^{\infty}{g(n\Delta t)e^{-2\pi ift}} \label{eqn:dft-01}
    			\end{align}
    			In case of a finite number $N$ of samples, we can express $g(n\Delta t)=g_n$ and so equation (\ref{eqn:dft-01}) becomes
    			\begin{align}
    				\hat{g}_\text{obs}(f)&=\sum_{n=0}^{N}{g_ne^{-2\pi ifn\Delta t}} \label{eqn:dft-02}
    			\end{align}
    			As per the Nyquist limit, the frequency range of relevance is $0\leqslant f\leqslant\dfrac{1}{\Delta t}$. This range can be evenly spaced into $N$ intervals with
    			\begin{align*}
    				\Delta f&=\dfrac{1/{\Delta t}}{N}=\dfrac{1}{N\Delta t} \\
    				\implies\Delta f\Delta t&=\dfrac{1}{N}
    			\end{align*}
    			Let the frequency range be spaced by $\Delta f$, i.e. $f=k\Delta f$, where $k$ is an integer. Then from equation (\ref{eqn:dft-02}), we obtain
    			\begin{align}
    				\hat{g}_\text{obs}(k\Delta f)&=\sum_{n=0}^{N}{g_ne^{-2\pi ik\Delta fn\Delta t}} \nonumber \\
    				\implies \hat{g}_k&=\sum_{n=0}^{N}{g_ne^{-2\pi ikn/N}} \label{eqn:dft-03}
    			\end{align}
    			Here, equation (\ref{eqn:dft-03}) gives the expression for the implementation of the discrete Fourier transform.
    			
    			From the DFT in equation (\ref{eqn:dft-03}), we can compute the \textit{classical periodogram} or \textit{Schuster periodogram} \cite{schuster1898investigation} as $\dfrac{1}{N}$ times the Fourier power spectrum for a continuous signal observed with uniform sampling using a Dirac comb:
    			\begin{align}
    				P_s(f)=\dfrac{1}{N}\left| \sum_{n=1}^{N}{g_ne^{-2\pi i ft_n}} \right|^2 \label{eqn:schuster-periodogram}
    			\end{align}
    			It must be clarified here that the periodogram is a statistic which is computed from the data and it is used to estimate the power spectrum of the underlying continuous signal of interest.
    			
    			In real-world observational fields such as astronomy, data collection is often subjected to a range of constraints and variability, leading to non-uniform sampling. Unlike controlled experiments where conditions can be held constant, astronomical observations by space-based observatories are influenced by a variety of factors:
    			\begin{enumerate}[i.]
    				\item \textit{Orbital constraints}: Space-based telescopes are often in specific orbits around the Earth, the Moon, or even the Sun. These orbits dictate observation windows and create periods when the telescope cannot point at certain objects due to the Sun, Earth, or other celestial bodies obstructing the line of sight.
    				\item \textit{Sun avoidance regions}: To prevent damage to sensitive instruments, space-based telescopes typically have sun avoidance zones where they cannot observe. This limits the ability to track certain objects continuously, leading to gaps in data collection.
    				\item \textit{Limited data storage and downlink bandwidth}: Space telescopes have finite onboard data storage and are dependent on ground stations for data downlink. Periods when the telescope is out of communication range or when storage reaches capacity can result in interruptions in data collection.
    				\item \textit{Thermal constraints and cooling periods}: Space-based telescopes equipped with infrared sensors or other heat-sensitive instruments may require cooling periods or specific thermal management strategies. This can restrict observation times and introduce non-uniform sampling in the data.
    				\item \textit{Instrument maintenance and calibration}: Routine instrument calibration and maintenance activities can lead to temporary suspension of scientific observations, causing gaps in data collection.
    			\end{enumerate}
    			These factors create unevenly spaced data points that pose significant challenges for the discrete Fourier transform, which is built on the assumption of uniform sampling intervals.
    			
    			\paragraph{Non-uniform sampling of data:}
    			In case the observed data is a signal that is sampled over a set of $N$ times, i.e. $\{t_n\}$, and so the window function my be expressed as
    			\begin{align}
    				W_{\{t_n\}}(t)=\sum_{n=1}^{N}{\delta(t-t_n)} \label{eqn:nonunif-dirac-comb}
    			\end{align}
    			In this scenario, upon applying the window function to the underlying continuous signal, the observed signal is obtained to be of the form:
    			\begin{align}
    				g_\text{obs}(t)&=g(t)W_{\{t_n\}}(t) \nonumber \\
    				\implies g_\text{obs}(t)&=\sum_{n=1}^{N}{g(t)\delta(t-t_n)} \label{eqn:obs-func-nonunif}
    			\end{align}
    			The Fourier transform of the observed signal is obtained using the convolution theorem to be as follows
    			\begin{align}
    				\mathscr{F}\{g_\text{obs}\}=\mathscr{F}\{g(t)\}*\mathscr{F}\{W_{\{t_n\}}(t)\} \label{eqn:FT-obs-func-nonunif}
    			\end{align}
    			However, in this case, the Fourier transform of the window function $W_{\{t_n\}}(t)$ would not be a Dirac comb. Rather, it would be noisier because the observation times have undergone randomization. Even though this noise may be reduced to a certain degree by increasing the number of observations, it would still persist. Also, increasing the number of observations might not be practically feasible in situation, such as in astronomy.
    				
    		\subsubsection{Mathematical Formulation of Lomb-Scargle Periodogram}
    			Using Euler's formula, the classical periodogram given in equation (\ref{eqn:schuster-periodogram}) can be re-cast in the following manner:
    			\begin{align}
    				P(f)&=\dfrac{1}{N}\left| \sum_{n=1}^{N}{g_ne^{-2\pi i ft_n}} \right|^2 \nonumber \\
    				\implies P(f)&=\dfrac{1}{N}\left[ \left\lbrace \sum_{n=1}^{N}{g_n\cos{(2\pi ft_n)}} \right\rbrace^2 + \left\lbrace \sum_{n=1}^{N}{g_n\sin{(2\pi ft_n)}} \right\rbrace^2 \right] \label{eqn:lomb-scargle-01}
    			\end{align}
    			%When the classical periodogram is applied to uniformly-sampled Gaussian noise, the resulting periodogram becomes a $\chi^2$ distribution. However, this does not hold good when the sampling becomes nonuniform, with the end result being that the periodogram distribution cannot, in general, be analytically expressed.
    			When the classical periodogram is applied to uniformly-sampled Gaussian noise, its distribution converges asymptotically to a $\chi^2$ distribution with known degrees of freedom, this property being contingent upon the assumption of uniform sampling. For non-uniformly sampled data, the distribution of the periodogram becomes considerably more complex, and a closed-form analytical expression is generally intractable.
    			
    			To address this issue, Scargle generalized the periodogram in equation (\ref{eqn:lomb-scargle-01}) as follows \cite{scargle1982studies}:
    			\begin{align}
    				P(f)&=\dfrac{A^2}{2}\left\lbrace \sum_{n=1}^{N}{g_n\cos{[2\pi f(t_n-\tau)]}} \right\rbrace^2 + \dfrac{B^2}{2}\left\lbrace \sum_{n=1}^{N}{g_n\sin{[2\pi f(t_n-\tau)]}} \right\rbrace^2 \label{eqn:lomb-scargle-02}
    			\end{align}
    			where $A$, $B$ and $\tau$ are arbitrary functions of the frequencies $f$ and the observing times $\{t_i\}$. Scargle also showed that one can choose unique forms of $A$, $B$ and $\tau$ in a manner such that we have the periodogram in equation (\ref{eqn:lomb-scargle-02})
    			\begin{enumerate}[a)]
    				\item reducing to the classical form in equation (\ref{eqn:schuster-periodogram}) in case the observations are uniformly sampled.
    				\item having statistics which are analytically computable.
    				\item becoming insensitive to global time-shifts in data.
    			\end{enumerate}
    			The functions of $A$, $B$ and $\tau$ that provide the above properties are
    			\begin{align*}
    				A&=\dfrac{1}{\sqrt{\sum_{n}{\cos^2{[2\pi f(t_n-\tau)]}}}} \\
    				B&=\dfrac{1}{\sqrt{\sum_{n}{\sin^2{[2\pi f(t_n-\tau)]}}}} \\
    				\tau&=\dfrac{1}{4\pi f}\tan^{-1}{\left\lbrace \dfrac{\sum_{n}{\sin{[4\pi ft_n)]}}}{\sum_{n}{\cos{[4\pi ft_n)]}}} \right\rbrace}
    			\end{align*}
    			In the above, the function $\tau$ ensures the time-shift invariance of the Lomb-Scargle periodogram. A notable feature of the Lomb-Scargle periodogram is its equivalence to the result obtained from fitting a simple sinusoidal model to the data at each frequency $f$ and constructing a periodogram from the $\chi^2$ goodness-of-fit statistic at each frequency. This approach allows the periodogram to be viewed as a measure of how well a sine wave of a given frequency fits the observational data. In this context, the introduction of the $\tau$ shift is crucial, as it serves to orthogonalize the normal equations used in the least squares fitting process. This orthogonalization ensures that the fitting procedure is unbiased and more computationally stable. This deep connection between traditional Fourier analysis and least-squares analysis has led to the widespread adoption of the term \textit{Lomb-Scargle periodogram} to describe this methodology, even though variants of this approach were utilized prior to Lomb and Scargle's formalization.
    			
    			%The similarity between the classical Fourier-based periodograms and the Lomb-Scargle periodograms implies that much of the theoretical reasoning developed for the former is also qualitatively applicable to the latter. For instance, the influence of window functions, which modify the observed power spectrum due to finite observation times, is still relevant when interpreting the results of the Lomb-Scargle periodogram, although quantitative predictions may differ. However, a key limitation of the standard Lomb-Scargle formulation is its reliance on the assumption that observational noise is uncorrelated white noise. In cases where the data contain more complex noise characteristics, such as coloured noise or correlated observational errors, the statistical guarantees provided by the Lomb-Scargle method may no longer hold, necessitating more advanced modifications or alternative approaches to accurately interpret the variability signals present in the data.
    		
    		\subsubsection{Uncertainties in Periodogram Results}
    			A critical aspect of reporting results derived from the Lomb-Scargle periodogram is the quantification of uncertainty associated with the estimated period. While traditional approaches often rely on error bars to express uncertainty, such a metric may be less meaningful in the context of Lomb-Scargle periodograms. The primary concern with periodograms is often the presence of false peaks or aliases, rather than a continuous distribution of potential period values around a true estimate. This disjointed nature of uncertainty necessitates a more nuanced approach to quantifying the reliability of the estimated period.
    			
    			\paragraph{Peak width and frequency precision:}
    			The presence of a peak within the Lomb-Scargle periodogram, characterized by its width and height, signifies the existence of a periodic signal. In the Fourier domain, the precision with which a peak's frequency can be identified is directly correlated with its width. A common metric for quantifying this width is the half-width at half-maximum, denoted by $f_{1/2}\approx T$.
    			
    			A more rigorous framework for assessing uncertainty can be established within the context of least squares analysis. In this interpretation, the inverse of the peak's curvature is directly related to the uncertainty in the estimated frequency. This perspective aligns with the Bayesian approach, which involves fitting a Gaussian curve to the exponentiated peak. The Bayesian uncertainty in the frequency estimate is influenced by both the number of samples $N$, and the average signal-to-noise ratio $\Sigma$. The scaling of this uncertainty can be approximated by the standard deviation of the estimated frequency:
    			\begin{align}
    				\sigma_f\approx\sqrt{\dfrac{2}{N\Sigma^2}} \label{eqn:stddev-est-freq}
    			\end{align}
    			This dependence arise from the fact that the Bayesian uncertainty is linked to the width of the exponentiated periodogram, which in turn is influenced by the height of the peak $P_\text{max}$.
    			
    			\paragraph{False alarm probability:}
    			A more pertinent metric for assessing the uncertainty associated with Lomb-Scargle periodogram results is the relative height of the signal peak to the background noise. The spurious peaks that arise in the periodogram are influenced by both the sample size $N$, and the signal-to-noise ratio. For scenarios with limited data and low signal-to-noise, these spurious peaks can rival the height of the true signal peak, making it challenging to distinguish between genuine and chance-driven features.
    			
    			The False Alarm Probability (FAP) serves as a standard approach for quantifying the significance of a periodogram peak. It represents the probability that a dataset devoid of any true signal would, due to random fluctuations in the noise, produce a peak of comparable magnitude. Scargle (1982) demonstrated that for pure Gaussian noise at any given frequency $f_0$, the unnormalized Lomb-Scargle periodogram statistic $Z=P(f_0)$, follows a $\chi^2$ distribution with two degrees of freedom. Consequently, the cumulative probability of observing a periodogram value less than $Z$ can be expressed as:
    			\begin{align}
    				P_\text{single}(Z)=1-\exp(-Z) \label{eqn:L-S-fap-01}
    			\end{align}
    			Equation (\ref{eqn:L-S-fap-01}) provides a quantitative measure of the likelihood that a given peak is due to chance.
    			
    			The FAP measures the likelihood that a peak in the periodogram is due to chance, rather than a genuine periodic signal. A low FAP means that the observed peak is unlikely to be a false alarm. Therefore, for a periodic signal to be considered significant, the corresponding FAP should be small. This suggests that the peak is highly likely to represent a real periodic component within the data.
    			
    			\paragraph{The Baluev method:}
    			Baluev (2008) introduced a refined approximation for the False Alarm Probability (FAP) in the context of Lomb-Scargle periodograms \cite{baluev2008assessing}. This approximation, given by:
    			\begin{align}
    				\text{FAP}(z)\approx 1 - P_\text{single}(z)e^{-\tau(z)}\label{eqn:L-S-fap-baluev-01}
    			\end{align}
    			provides a conservative upper bound, even for time series characterized by highly structured window functions. Here, $P_\text{single}$ denotes the cumulative probability of observing a periodogram value less than $Z$ under the null hypothesis, while $\tau(z)$ is a correction factor defined as $\tau(z)\approx W(1 - z)^{(N-4)/2}\sqrt{z}$, where $W$ represents the effective width of the observing window in units of the maximum frequency and $N$ is the total number of data points.
    			
    			While this approximation is not exact, it offers a reliable upper bound for alias-free periodograms. Notably, it maintains a reasonable degree of accuracy even for more realistic survey windows that exhibit non-trivial structure. The Baluev method has become the standard algorithm for computing the FAP in many software implementations of the Lomb-Scargle periodogram.
    		
    	\subsection{Implementation of Lomb-Scargle Periodogram}
    		Here we outline the methodology we have employed to investigate the variability characteristics of lightcurves extracted from SSS across two energy ranges: 0.2-1.0 keV and 0.2-12.0 keV. The analysis leverages the Lomb-Scargle periodogram to identify and characterize periodic signals within the lightcurves. By systematically applying these steps to the lightcurves of SSS at the two specified energy ranges, this methodology enables a comprehensive characterization of their variability patterns. The combination of Lomb-Scargle periodogram analysis, statistical significance evaluation, phase-folding, and sinusoidal fitting provides a robust framework for uncovering periodic signals and quantifying their properties within the lightcurve data.
    		    	
    		\subsubsection{Data Acquisition and Preprocessing}
    			\begin{enumerate}[I.]
    				\item Google Drive is mounted within Google Colab to facilitate seamless access to the relevant FITS files containing the lightcurve data.
    				\item The \texttt{astropy.io} module's \texttt{fits.open()} function is used to open the FITS files and extract the lightcurve data.
    				\item The extracted lightcurve information is converted into \texttt{NumPy} arrays for efficient numerical operations.
    				\item Visual inspection of the lightcurve is performed to identify potential outliers. These outliers, if present, are removed using \texttt{NumPy} masks to ensure they don't influence the subsequent analysis. The cleaned lightcurve is then visualized.
    			\end{enumerate}
    		
    		\subsubsection{Timing Analysis using Lomb-Scargle Periodogram}
    			\begin{enumerate}[I.]
    				\item A \texttt{LombScargle} object is created using the \texttt{astropy.timeseries} module. This object encapsulates the time and rate arrays extracted from the lightcurve to compute the Lomb-Scargle periodogram.
    				\item The calculated Lomb-Scargle periodogram is utilized to retrieve the corresponding frequency and power arrays using the \texttt{autopower()} method of the \texttt{LombScargle} object.
    				\item The Lomb-Scargle periodogram is displayed graphically. Prominent peaks are visually identified for further analysis.
    				\item Peaks corresponding to harmonics associated with the observation window and the time bin size are disregarded, as they do not represent genuine periodicities within the lightcurve data. Rather these are artefacts due to the observation cadence and the time binning process.
    			\end{enumerate}
    		
    		\subsubsection{Statistical Significance and Phase-Folded Lightcurves}
    			\begin{enumerate}[I.]
    				\item The \texttt{false\_alarm\_probability()} method of the \texttt{LombScargle} object is employed to compute the FAP for the identified peaks. This metric quantifies the probability of observing such peaks arising purely from random noise.
    				\item Employing the periods corresponding to the identified significant peaks, phase-folded lightcurves are constructed. These lightcurves represent the binned data points as a function of their phase within a complete cycle (0 to $2\pi$ radians).
    				\item The \texttt{curve\_fit()} function from the \texttt{scipy.optimize} module is used to fit a sinusoidal function of the following form to each phase-folded lightcurve:
    				$$C=A\sin(2\pi\nu t+\phi)+C_0$$
    				Here, $A$ represents the amplitude, $\nu$ the frequency, $\phi$ the phase offset, and $C_0$ the DC offset. The best-fit values for these parameters are obtained through the fitting process.
    			\end{enumerate}
    		
    		\subsubsection{Visualization and Result Presentation}
    			\begin{enumerate}[I.]
    				\item The phase-folded lightcurve along with the superimposed best-fit sinusoid are plotted to visually assess if the calculated variability aligns with the observed data points.
    				\item The analysis results are consolidated into a table, including the following:
    				\begin{enumerate}[i.]
    					\item Peak frequencies (in mHz)
    					\item Corresponding periods (in ks)
    					\item Power densities
    					\item False alarm probabilities (FAP)
    					\item Best-fit parameters from sinusoidal fitting ($A$, $\phi$, $C_0$)
    					\item Amplitude-to-mean ratio (AMR): Calculated as $(A/C_0)\times 100\%$, indicating the strength of the periodic signal relative to the mean lightcurve level.
    				\end{enumerate}
    			\end{enumerate}