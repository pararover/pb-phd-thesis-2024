\chapter*{ABSTRACT}
    
    This thesis presents a comprehensive study of Supersoft X-ray Sources (SSS), utilizing advanced spectral fitting techniques and variability analysis. By analyzing high-resolution X-ray spectra from multiple observatories, we have developed robust models to characterize the spectral properties and physical processes governing SSS. The study explores the diversity of SSS types, their spectral features, luminosity, and variability characteristics. Key findings include differences in interstellar absorption between Milky Way and Large Magellanic Cloud SSS, a correlation between luminosity and mass accretion rate, and the presence of periodic modulations. These results provide insights into the evolutionary pathways of SSS and their role in astrophysical phenomena. Future research directions include expanding the dataset, refining theoretical models, and conducting high-resolution timing studies to further unravel the complexities of these fascinating objects.
    
    \textbf{Keywords:} X-rays: binaries, X-rays: individual: CAL 83, X-rays: individual: RS Oph, X-rays: individual: RX J0019.8+2156, X-rays: individual: RX J0527.8-6954, X-rays: individual: RX J0925.7-4758, methods: observational, techniques: spectroscopic, techniques: photometric