\chapter{MULTI-OBSERVATORY SPECTRAL ANALYSIS OF SUPERSOFT X-RAY SOURCES} \label{chap:multi-obs}
    %\doublespacing
    \minitoc
    \emph{Abstract of chapter \ref{chap:multi-obs}}
    
    \section{Literature Review} \label{multi-obs:lit-rev}
    	Supersoft X-ray sources (SSS) represent an important class of celestial objects. They were initially recognized as a distinct class of intrinsically luminous X-ray sources by Trümper \textit{et al.} \cite{trumper1991x}, Greiner \textit{et al.} \cite{greiner1991rosat}, and Kahabka \textit{et al.} \cite{kahabka97}. These sources are classified based on their X-ray luminosities, which are typically around the Eddington limit ($\sim 10^{38}$ erg s$^{-1}$), indicating their exceptionally high brightness. However, their defining feature is their extremely soft X-ray spectra, with weak or negligible emission beyond $\sim 1$ keV and effective blackbody temperature no more than $\sim 100$ eV (or $\lesssim 1200$ kK) \cite{kahabka06}. The understanding of the spectral characteristics of SSS promises to pave the way for a deeper exploration of their role in the broader astrophysical landscape.
    
    \section{Journal of Observations} \label{multi-obs:journal}
    
    \section{Data Reduction and Analysis} \label{multi-obs:red-analysis}
    
    	\subsection{XMM-Newton EPIC-pn Data Reduction} \label{multi-obs:red-analysis:epic-pn}
    	
    	\subsection{ASCA SIS1 Data Reduction} \label{multi-obs:red-analysis:sis1}
    	
    	\subsection{Chandra ACIS Data Reduction} \label{multi-obs:red-analysis:acis}
    	
    	\subsection{NICER XTI Data Reduction} \label{multi-obs:red-analysis:xti}
    	
    \section{Results} \label{multi-obs:results}
    
    	\subsection{Observed Count Rates} \label{multi-obs:results:count-rates}
    	
    	\subsection{NLTE Continuum Model} \label{multi-obs:results:nlte}
    	
    	\subsection{Unfolded Spectra} \label{multi-obs:results:unfolded}
    	
    	\subsection{Luminosity versus Effective Temperature} \label{multi-obs:results:L-vs-Teff}
    	
    	\subsection{Presence of Elemental Absorption Edges} \label{multi-obs:results:abs-edge}
    	
    \section{Discussion} \label{multi-obs:discussion}
    
    	\subsection{Best-fit Continuum Model} \label{multi-obs:discussion:cont-mod}
    	
    	\subsection{NLTE Pure H Model Atmosphere} \label{multi-obs:discussion:nlte-pureH}
    	
    		\subsubsection{Structural Equations}
    		
    		\subsubsection{Numerical Solution using ALI}
    	
    	\subsection{Inferences from Results} \label{multi-obs:discussion:inference}
    	
    	\subsection{Relative Strengths of Absorption Edges} \label{multi-obs:discussion:abs-edge-strength}