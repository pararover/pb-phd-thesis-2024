\def\baselinestretch{1}
\chapter{RESULTS} \label{chap:results}
    \minitoc
    \emph{Abstract of chapter \ref{chap:results}}
    
    \section{SPECTRAL LINE IDENTIFICATION}
    
    \section{XMM-NEWTON RGS SPECTRAL FIT}
    The best fit model, i.e. M11, contains two different additive components that simulate optically thin plasma. One of them, i.e. \texttt{apec} uses line lists from the AtomDB database, while the other, i.e. \texttt{mekal} which is based on calculations by Mewe, Kaastra and Liedahl \cite{meka,liedahl}. The latter could indicate the presence of a hot corona. The presence of the \texttt{rauch} model component in the best fit model suggests a substantial contribution due to an NLTE stellar atmosphere. The model also consists of the \texttt{swind1} component, which simulates the presence of a stellar wind component along the line-of-sight. The presence of all the above model components vindicates our hypothesis of the inclusion of non-steady states, NLTE and stellar winds in the radiative processes in X-ray binary systems such as those of RX J0925.7-4758.
    
    \section{P CYGNI PROFILING}