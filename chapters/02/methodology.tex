\chapter{METHODOLOGY} \label{chap:methodology}
    %\doublespacing
    \minitoc
    
    \newpage
    \begin{center}
    	\emph{Abstract of chapter \ref{chap:methodology}}
    \end{center}
    
    Insert abstract here.
    
    \section{Spectral Study of Supersoft X-ray Sources} \label{methodology:spectral}
    	
    	
    \section{Variability Study of Supersoft X-ray Sources} \label{methodology:variability}
    
    	\subsection{Introduction to Lomb-Scargle Periodogram}
    		Introduce the concept of periodograms, emphasizing their role in time series analysis for detecting periodic signals.
    		
    		\subsubsection{Historical Background}
    		
    		\subsubsection{Importance in Astrophysics}
    	
    	\subsection{Theory of Lomb-Scargle Periodogram}
    	
    		\subsubsection{Limitation of Discrete Fourier Transform}
    			Suppose that $g(t)$ is a \textit{signal function} and $W(t)$ represents the observation \textit{window function}, then the observed signal is represented by the following function:
    			\begin{align}
    				g_\text{obs}(t)=g(t)W(t) \label{eqn:obs-func}
    			\end{align}
    			Then, by the Fourier convolution theorem, we have,
    			\begin{align}
    				\mathscr{F}\left\lbrace g_\text{obs} \right\rbrace = \mathscr{F}\left\lbrace g(t) \right\rbrace * \mathscr{F}\left\lbrace W(t) \right\rbrace
    			\end{align}
    			In the case of the discrete Fourier transform, where the continuous signal is sampled at regular intervals, the window function becomes a Dirac comb, which is a regular grid of Dirac delta functions with spacing of $\Delta t$, i.e.
    			\begin{align}
    				W(t)=\text{III}_{\Delta t}(t)=\sum_{n=-\infty}^{\infty}{\delta(t-n\Delta t)} \label{eqn:dirac-comb}
    			\end{align}
    			
    			Hence, equation \ref{eqn:obs-func} becomes
    			\begin{align}
    				g_\text{obs}(t)=g(t)\text{III}_{\Delta t}(t) \label{eqn:obs-func:dft}
    			\end{align}
    			Taking the Fourier transform of equation (\ref{eqn:obs-func:dft}), we can have
    			\begin{align}
    				\hat{g}_\text{obs}(f)&=\FT{g_\text{obs}(t)}{f} \nonumber \\
    					&=\FT{g(t)\text{III}_{\Delta t}(t)}{f} \nonumber \\
    					&=\FT{g(t)\left\lbrace \sum_{n=-\infty}^{\infty}{\delta(t-n\Delta t)} \right\rbrace}{f} \nonumber \\
    					&=\sum_{n=-\infty}^{\infty}{\int_{-\infty}^{\infty}{\delta(t-n\Delta t)g(t)e^{-2\pi ift}\diff{t}}} \nonumber \\
    				\implies\hat{g}_\text{obs}(f)&=\sum_{n=-\infty}^{\infty}{g(n\Delta t)e^{-2\pi ift}} \label{eqn:dft-01}
    			\end{align}
    			In case of a finite number $N$ of samples, we can express $g(n\Delta t)=g_n$ and so equation (\ref{eqn:dft-01}) becomes
    			\begin{align}
    				\hat{g}_\text{obs}(f)&=\sum_{n=0}^{N}{g_ne^{-2\pi ifn\Delta t}} \label{eqn:dft-02}
    			\end{align}
    			As per the Nyquist limit, the frequency range of relevance is $0\leqslant f\leqslant\dfrac{1}{\Delta t}$. This range can be evenly spaced into $N$ intervals with
    			\begin{align*}
    				\Delta f&=\dfrac{1/{\Delta t}}{N}=\dfrac{1}{N\Delta t} \\
    				\implies\Delta f\Delta t&=\dfrac{1}{N}
    			\end{align*}
    			Let the frequency range be spaced by $\Delta f$, i.e. $f=k\Delta f$, where $k$ is an integer. Then from equation (\ref{eqn:dft-02}), we obtain
    			\begin{align}
    				\hat{g}_\text{obs}(k\Delta f)&=\sum_{n=0}^{N}{g_ne^{-2\pi ik\Delta fn\Delta t}} \nonumber \\
    				\implies \hat{g}_k&=\sum_{n=0}^{N}{g_ne^{-2\pi ikn/N}} \label{eqn:dft-03}
    			\end{align}
    			Here, equation (\ref{eqn:dft-03}) gives the expression for the implementation of the discrete Fourier transform.
    			
    			From the DFT in equation (\ref{eqn:dft-03}), we can compute the \textit{classical periodogram} or \textit{Schuster periodogram} \cite{schuster1898investigation} as $\dfrac{1}{N}$ times the Fourier power spectrum for a continuous signal observed with uniform sampling using a Dirac comb:
    			\begin{align}
    				P_s(f)=\dfrac{1}{N}\left| \sum_{n=1}^{N}{g_ne^{-2\pi i ft_n}} \right|^2 \label{eqn:schuster-periodogram}
    			\end{align}
    		
    		\subsubsection{Mathematical Formulation of Lomb-Scargle Periodogram}
    		
    		\subsubsection{Statistical Properties}
    	
    	\subsection{Implementation of Lomb-Scargle Periodogram}
    	
    		\subsubsection{Computational Techniques}
    		
    		\subsubsection{Software Tools and Libraries}
    	