\newpage
\begin{center}
	%\emph{Abstract of chapter \ref{chap:tool}}
	\emph{Abstract}
\end{center}

    The identification of absorption and emission lines due to atomic species in the spectra of a X-ray binaries can reveal a wealth of information regarding the composition and physics of the stellar atmosphere. With the availability of high-resolution X-ray spectra from the RGS equipment on-board the XMM-Newton satellite, the study of such lines offers valuable diagnostics into the behaviour and evolution of the source object. Currently, data related to various atomic transitions, which lead to line formation, are made publicly available in the form of credible databases, one of them being the Atomic Spectra Database (ASD) at the National Institute of Standards and Technology (NIST). In this chapter, we present the development of a Python-based tool that accesses the relevant atomic data at NIST ASD for a given set of atomic species in a specific wavelength range and then overlays these lines on top of an X-ray spectrum obtained by the RGS spectrum of the XMM-Newton. With this tool, the astronomer can perform the important preliminary task of rudimentary line identification, before proceeding to advanced analysis of the X-ray data.